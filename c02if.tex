%!TEX program = xelatex
\documentclass{article}
\usepackage[a5paper,hmargin=17mm,tmargin=15mm,bmargin=25mm]{geometry}

\usepackage{ifxetex}
\ifxetex
 \usepackage{fontspec}
 \setmainfont[Scale=1.1]{Arno Pro}
 \setmonofont[Scale=.95]{Consolas}
 \usepackage{unicode-math}              %% пакет для загрузки шрифтов математического режима 
 \setmathfont{[latinmodern-math.otf]}
 \setmathfont[range=\mathit/{latin,Latin}]{Arno Pro Italic}
 \setmathfont[range=up]{Arno Pro}
\else
 \usepackage[utf8]{inputenc}
\fi
\usepackage[russian]{babel}
\usepackage{enumitem, minted, pdfpages}


\begin{document}
\section*{{\normalsize Лабораторная работа 2.} \\Ветвления}

Цель этой лабораторной работы~--- овладеть понятием ветвления и научиться записывать ветвления в языке Си. 

\bigskip
\hfill\textbf{ОБЩИЕ ЗАДАНИЯ}\hfill{~}
\smallskip
\sloppy

\begin{enumerate}
\item
Напишите программу, которая вводит с клавиатуры одно целое число, и выводит \texttt{POL}, если оно положительно, \texttt{OTR}~--- если отрицательно, и \texttt{NUL}, если это нуль.

\item 
Напишите программу, которая вводит с клавиатуры один символ, и выводит \texttt{DIGIT}, этот символ является цифрой, \texttt{CAPITAL}~--- если заглавной латинской буквой, \texttt{LOWERCASE}~--- если строчной, и выводит \texttt{NON-ALPHANUMERIC} в противном случае.

\item
Введите с клавиатуры год в интервале от 1582 до 2200, выведите \texttt{LEAP}, если он високосный, или \texttt{NORMAL}, если нет. Выведите \texttt{ERROR}, если номер года больше 2200 или меньше 1582. (К примеру, 2019 год~--- не високосный, 2020~--- високосный, 1900 и 2100~---не високосные, 2000~--- високосный.)

\item
Напишите программу, которая вводит с клавиатуры возраст $n$ лет и выводит сообщение \texttt{ВАМ $n$ ЛЕТ/ГОДА/ГОД}, используя правильное слово, если $1 \leqslant n \leqslant 100$, или  \texttt{ERROR} в противном случае 
(используйте \texttt{setlocale("ru"{}, LC\_ALL);}).

\item
(\texttt{switch/case}) Введите с клавиатуры число $n$ и выведите английское название соответствующей цифры от 0 до 9 заглавными буквами (\texttt{ZERO}, \texttt{ONE}, \ldots \texttt{NINE}), либо \texttt{ERROR}, если число меньше 0 или больше 9. Массивы не использовать, \texttt{if} не использовать.
\end{enumerate}
\end{document}